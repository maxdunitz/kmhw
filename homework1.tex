% --------------------------------------------------------------
% This is all preamble stuff that you don't have to worry about.
% Head down to where it says "Start here"
% --------------------------------------------------------------
 
\documentclass[12pt]{article}
 
\usepackage[margin=1in]{geometry} 
\usepackage{amsmath,amsthm,amssymb}
 
\newcommand{\N}{\mathbb{N}}
\newcommand{\Z}{\mathbb{Z}}
 
\newenvironment{theorem}[2][Theorem]{\begin{trivlist}
\item[\hskip \labelsep {\bfseries #1}\hskip \labelsep {\bfseries #2.}]}{\end{trivlist}}
\newenvironment{lemma}[2][Lemma]{\begin{trivlist}
\item[\hskip \labelsep {\bfseries #1}\hskip \labelsep {\bfseries #2.}]}{\end{trivlist}}
\newenvironment{exercise}[2][Exercise]{\begin{trivlist}
\item[\hskip \labelsep {\bfseries #1}\hskip \labelsep {\bfseries #2.}]}{\end{trivlist}}
\newenvironment{problem}[2][Problem]{\begin{trivlist}
\item[\hskip \labelsep {\bfseries #1}\hskip \labelsep {\bfseries #2.}]}{\end{trivlist}}
\newenvironment{question}[2][Question]{\begin{trivlist}
\item[\hskip \labelsep {\bfseries #1}\hskip \labelsep {\bfseries #2.}]}{\end{trivlist}}
\newenvironment{corollary}[2][Corollary]{\begin{trivlist}
\item[\hskip \labelsep {\bfseries #1}\hskip \labelsep {\bfseries #2.}]}{\end{trivlist}}
 
\begin{document}
 
% --------------------------------------------------------------
%                         Start here
% --------------------------------------------------------------
 
\title{Homework}%replace X with the appropriate number
\author{Max Dunitz\\ %replace with your name
Kernel Methods} %if necessary, replace with your course title
 
\maketitle


\begin{exercise}{1} 
	\begin{enumerate}
		\item The symmetry of the kernel $K(x,y) = \cos (x-y) = \cos (y-x) = K(y,x)$ follows immediately from the evenness of the cosine function. The positive definiteness can be seen by noting that $K(x,y) = \cos(x-y) = \frac{1}{2} e^{j(x-y)} + \frac{1}{2} e^{-j(x-y)}$, a nonnegative linear combination of two exponential kernels. The positive definiteness of the latter is established by the following: For any real collection of $\alpha_i$, $\sum_i \sum_k \alpha_i \alpha_k e^{j(x_i - y_k)} = \big(\sum_i \alpha_i e^{jx_i}\big)\big(\sum_k \alpha_k e^{-j x_i} \big).$ Letting $s=\sum_i \alpha_i e^{jx_i}$, this becomes $s \cdot \bar{s} = |s|^2 \geq 0$, thus establishing the positive definiteness of $K_{exp} = e^{j(x-y)}$. The argument for the positive definiteness of $K_{exp-conjugate} = e^{-j(x-y)}$ holds by symmetry. Thus this cosine kernel, as the nonnegative combination of two positive definite kernels, is positive definite.
		\item The symmetry of the kernel $K(x,y) = \frac{1}{1-x^Ty} = \frac{1}{1-y^Tx} = K(y,x)$ follows immediately from the symmetry of the linear kernel $K_{lin}(x,y) = x^Ty$. Since $\mathcal{X}$ contains only those vectors $x \in \mathbb{R}^p$ for which $||x||_2 < 1$, by the Cauchy-Schwartz inequality we have that for $x,y \in \mathcal{X}, |x^T y| < 1$ and so $K(x,y)$ may be expressed as the convergent infinite series $1 + (x^T y) + (x^T y)^2 + (x^T y)^3 + \ldots.$ We can thus define a sequence of positive definite kernels $K_i(x,y) = \sum_{k=0}^i (x^Ty)^k$ for $i >0$ that converges pointwise to $K(x,y) = \frac{1}{1-x^Ty}$. Each partial sum in the sequence is positive definite as it is a non-weighted sum of positive definite (polynomial) kernels. But there is a slight complication: the zeroth-order kernel $K(x,y) = 1$ is only positive semidefinite, not positive definite. That is, the rank-one Gram matrix associated with the symmetric ``semidefinite kernel" $K(x,y) = 1$---ie, the all-ones matrix---has one positive eigenvalue but the rest are zero. This is not a concern as the sum of a positive definite matrix with that of a positive semidefinite matrix is positive definite: if $x^TAx > 0$ and $x^TBx \geq 0$ then $x^T(A+B)x = x^TAx + x^T B x > 0.$
		\item Since $P(A \cap B) = P(B \cap A)$ and $P(A)P(B) = P(B)P(A)$, the kernel is symmetric. Given a set $A$ define the indicator function $I_A: \mathbb{R} \rightarrow \mathbb{R}$ that maps $x$ to 1 if $x\in A$ and 0 otherwise. This function is measurable when $A$ is measurable with respect to a probability measure. Let $\mu_A \triangleq \mathbb{E}[I_A] = \int I_A (x) d \mu (x) = \int_{x \in A} d \mu (x) = P(A).$ Now define the function $\phi_A(x) = I_A(x) - \mu_A.$ We can repeat this process, making equivalent definitions of indicator function, mean, and difference function $\phi$, for any set---and use them to rewrite our kernel as an inner product: $K(A,B) = P(A\cap B) - P(A)P(B) = \int I_A (x) I_B(x) d \mu(x) - \big(\int I_A(x) d \mu(x) \big) \cdot \big(\int I_B(x) d\mu(x)\big) = \mathbb{E}[I_A I_B] - \mathbb{E}[I_A]\mathbb{E}[I_B] =\mathbb{E}[I_AI_B] - \mu_A\mu_B = \mathbb{E}[I_AI_B] - \mu_A\mathbb{E}[I_B] - \mu_B \mathbb{E}[I_A] -\mu_A\mu_B = \mathbb{E}[(I_A -\mu_A)(I_B - \mu_B)] = \int \phi_A (x) \phi_B (x) d \mu (x) = \langle \phi_A, \phi_B \rangle_{L_2(\mu)}$. Having found an appropriate inner product, we can apply the Aronsajn theorem, thereby establishing the positive definiteness of $K(A,B)$. We note that for all $x\in \mathbb{R}$, we have that $I_A(x)I_B(x) \leq I_A(x)$ and $\int I_A(x) \cdot \mu_B d\mu(x) = \mu_B \cdot \int I_A(x) d\mu(x)$ is finite since $A$ an $B$ are measurable sets (and hence $\mu_A$ and $\mu_B$ are finite), this is all good and kosher.
		\item We note that $K_4(x,y) = \min (f(x)g(y), f(y)g(x)) = \min (f(y)g(x), f(x)g(y)) = K_4(y,x)$, where the second equality holds from the symmetry of the min function. Thus $K_4$ is symmetric. To see that this is a positive definite kernel consider the mapping $\phi: \mathbb{R}^+ \rightarrow \mathbb{R}^+$ given by $x \rightarrow (I_{[0,f(x)]}, I_{[0, g(x)]})$
		\item The symetry of the kernel follows immediately from the facts that $A \cup B = B \cup A$ and $A \cap B = B \cap A$. We saw in class that $K_{min-over-max}(x,y) = \frac{\min(x,y)}{\max(x,y)}$ is a positive definite kernel. Let $I_A(x)$ be the indicator for $A$ (ie, it equals 1 if $x \in A$ and 0 otherwise) and $I_B(x)$ the indicator for $B$. Then $K(A,B) = \sum_{x \in E} \frac{|A \cap B|}{|A \cup B|} = \sum_{x \in E} \frac{\min(I_A(x), I_B(x))}{\max(I_A(x), I_B(x))}$ is a nonnegative weighted sum of positive definite kernels and is therefore positive definite.
			
	\end{enumerate}

\end{exercise}


\end{document}
